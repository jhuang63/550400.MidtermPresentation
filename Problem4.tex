\documentclass[12pt,letterpaper]{article}

\usepackage{amsmath, amsthm, amssymb, amsfonts}
\usepackage{graphicx}
\usepackage{bm}
\usepackage{natbib}

\theoremstyle{definition}
\newtheorem{dfn}{Definition}

\begin{document}


\section{Formulate the Problem} 
Is the tennis game fair? The problem as stated is unanswerable. A better approach would be to provide a explicit definition of 'fair': whether being the receiver or not (for the first round), has not effect on the probability of winning the game for each individual players. Thus we can try to determine how players perform when they are receivers or senders for the first round. As long as there exists a noticeable difference between being receiver and being sender, we could then conclude that the tennis game is not fair.

We have changed the original problem considerably. We are going to answer how players perform when they are receivers and senders in the first round. Actually then questions need further refinement. For example, different players have different strengths and weaknesses, and their strengths of being receivers will affect the research conclusion. The question how player perform as receiver contains a trap, because it invites us to ignore these variations. However, the best idea is probably to proceed and to realize that in studying a real situation we will eventually need to return to step1 and formulate the questions more precisely.

\section{Outline the Model}
The major factor that will explain how player performs as receiver is the probability of winning the round as a receiver. By analyzing the historical data of a specific player, we can obtain how probabilities vary within different rounds for a whole game and the overall probability P(A) for winning the round as a receiver. From this we can decide how the player should perform for the first round (as a receiver), and we use this overall probability P(A) as a benchmark. After that, we should calculate the P(B) -probability of winning the first round as the receiver, from the historical data. 
  
\section{Is It Useful}
From this

\section{Test the Model}
We've complete the steps 1, 2 and 3 for the first part of the problem. We don't have the data to carry out step4, but it should be relatively straightforward.




\end{document}
